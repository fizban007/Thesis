\begin{center}

{\Large \bf Abstract} \vskip.2in

{\Large \bf \thesistitle} \vskip.2in

{\Large  \thesisauthor} \vskip.2in

\end{center}

Neutron stars are surrounded by dense magnetospheres with nontrivial magnetic
field structure. They are sources of multi-band emission from radio waves to
very high energy gamma-rays. Pulsar wind nebulae observations also show that a
large number of $e^{\pm}$ pairs flow from the neutron star, which are produced in
the magnetosphere. The structure of the magnetosphere, the mechanism of pair
production and particle acceleration in the magnetosphere, and how magnetic
energy is converted to kinetic energy is a complex problem that only recently
has started to be addressed fully from first principles. In this dissertation I
describe how I developed a numerical code tailored to study this problem. A
detailed description of the code and method is given, then it is used to study
the pair discharge mechanism in the magnetosphere of rotating neutron stars
whose rotating axis is aligned with the magnetic axis. It was found that to form
the an active magnetosphere it is necessary to have pair creation all the way
towards the light cylinder. In the dissertation I classify the pulsars into two
classes, and describe their differences.

The magnetospheres of magnetars are believed to be different from ordinary
pulsars, in that they are sustained not by the rotation of the star, but by a twist
launched from the stellar surface due to some sudden breakdown of the crust. I
apply the same numerical tool to study the particle acceleration and pair
creation mechanism in the twisted magnetosphere of the magnetar, showing where the
gap is, and how the magnetosphere evolves over time. The magnetic twist was
found to live much longer than the Alfv\'en time of the system, and slowly
dissipates through developing a cavity in the inner magnetosphere. This not only
explains the long term evolution of the magnetar lightcurve after an outburst,
but also explains the observed evolution hotspots on the stellar surface.

% Local Variables:
% TeX-master: "../thesis"
% End: