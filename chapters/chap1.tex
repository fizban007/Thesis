\zotelo{../thesis.bib}

\chapter{Particle-in-Cell Simulation}
\label{chap:pic}

\section{Introduction}
\label{sec:introduction}

Particle-in-Cell(PIC) codes are a powerful tool to study the kinetic properties
of plasma from first principles. The strength of a PIC code is that it can
resolve the plasma skin depth, and reproduce the interaction between particles
and fields, making them invaluable in the study of particle acceleration
processes in plasmas. It was very successfully applied to collisionless shocks
and reconnection processes
%TODO: references
. However, because a PIC code has to resolve the
plasma skin depth which is usually many orders of magnitude smaller than the
scale of a realistic astrophysical problem, this kind of simulation is extremely
expensive and very often only applied to local problems to study the
microphysics in a relatively large system.

This chapter introduces the PIC technique in detail, and then introduces
Aperture, a versatile PIC code designed and developed from scratch as part of my
PhD thesis. The name is a recursive acronym which stands for Aperture is a code
for Particles, Electrodynamics and Radiative Transfer at Ultra-Relativistic
Energies. The original and main purpose of the code was to simulate the global
structure of the Pulsar magnetosphere from first principles. However, we
designed the code to be general enough to be applied to many different problems
in Astrophysics, especially in problems where particles are accelerated in
strong magnetic fields and capable of producing pair cascade.

In Section \ref{sec:particle-cell-method} of the paper we will present
the numerical algorithms and techniques employed in the Aperture code,
from coordinate systems to boundary conditions. In Section
\ref{sec:test-problems} we will present some simple test cases to show
the validity of the code. Finally in Section
\ref{sec:example-applications} we will present some of the problems
that we can address using this new PIC code.

\section{The Particle-in-Cell Method}
\label{sec:particle-cell-method}
% This section is the meat of this paper. There will probably be many
% subsections.
The PIC method is essentially a way to solve the Maxwell-Vlasov
equations by approximating the plasma distribution function using the
sum of a large number of discrete macro-particle distributions. The
system of equations under question is simply the Maxwell equations
combined with the Vlasov equation:
\begin{align}
    \frac{\partial f^{s}}{\partial t} + \mathbf{u}\cdot\frac{\partial f^s}{\partial \mathbf{x}} &+ \frac{q^s}{m^s}(\mathbf{E} + \mathbf{u}\times \mathbf{B})\cdot\frac{\partial f^s}{\partial (\gamma \mathbf{u})} = 0 \label{eq:vlasov}\\
    \nabla\cdot \mathbf{E} &= 4\pi\rho \\
    \nabla\cdot \mathbf{B} &= 0 \\
    \nabla\times \mathbf{E} &= -\frac{1}{c}\partial_t \mathbf{B} \\
    \nabla\times \mathbf{B} &= \frac{1}{c}\partial_t \mathbf{E} + \frac{4\pi}{c} \mathbf{j}
\end{align}
where $s$ denotes the particle species (electrons, positrons, ions,
\dots). We assume a collisionless plasma by setting the right hand
side of equation \eqref{eq:vlasov} to zero.

The charge and current densities are found by taking the moments of
the distribution function $f^s$ over the momentum space
\begin{align}
  \rho &= \int d \mathbf{u} \sum_s q^s f^s(\mathbf{x}, \mathbf{u}) \label{eqn:rho} \\
  \mathbf{j} &= \int d \mathbf{u} \sum_s q^s \mathbf{u} f^s(\mathbf{x}, \mathbf{u}) \label{eqn:j}
\end{align}
Macro particles are introduced to sample the distribution function
$f^s$ in both position and momentum space. A macro particle can be
viewed as a collection of particles smeared out in space, with
identical physical momentum $\mathbf{p}_{p}$. Therefore, each
particle represents a distribution function of the type
\begin{equation}
    \label{eq:single-particle}
    f^s_{p}(\mathbf{x}, \mathbf{u}) = \delta(\gamma \mathbf{u} - \mathbf{p}_{p}) S(\mathbf{x} - \mathbf{x}_{p})
\end{equation}
where $S$ is a function that describes the shape of the macro
particle, with the property that $S$ has finite support, and that the
integral of $S$ over all space is normalized to 1. The idea is that,
if each individual macro particle satisfies the Vlasov equation, then
the linear superposition of a large number of them still satisfy the
Vlasov equation, and should provide a good approximation for the
dynamics of the plasma.

The dynamic equations for the macro particles can be derived by taking
the moments of the Vlasov equation with the single particle
distribution function \eqref{eq:single-particle}. It turns out that
these dynamic equations are identical to the equations of motion of a
single particle in electromagnetic field:
\begin{align}
  \label{eq:particle-equations}
  \frac{d\mathbf{x}_p}{dt} = \mathbf{u}_p = \frac{\mathbf{p}_p}{\gamma_p} \\
  \frac{d\mathbf{p}_p}{dt} = \frac{q_p}{m_p}(\mathbf{E} + \mathbf{u}_p\times \mathbf{B})
\end{align}
Therefore it is justified to treat macro particles as their name
suggests: simply as physical particles. The code simply trace their
motion in the electromagnetic field as described by the above dynamic
equations.

\subsection{Discretization}
\label{sec:discretization}

To solve the Maxwell equations and particle dynamic equations
numerically, one needs to discretize to a finite grid, hence the name
Particle-in-Cell. The discretization is done on space and time, so
that fields $\mathbf{E}$ and $\mathbf{B}$ are sampled on a
finite grid, as well as the current and charge densities
$\mathbf{J}$ and $\rho$. One evolves the equations using a given
time evolution scheme step by step starting from the initial
condition. At each step, one updates the positions and momenta of all
particles according to the fields on the grid, computes the current
density due to particle motion, and uses this current density to
evolve the fields themselves.

Aperture is a 2.5D PIC code, meaning that the spatial grid is 2D,
while we compute all 3 components of the field quantities on the
grid. This is applicable when the problem has inherent symmetry, such
as axisymmetry or translational invariance in one direction. We use
the classical staggered Yee grid for electric and magnetic fields.
% TODO: Insert a picture of the Yee grid

Since particles stream freely in the cells, the electric and magnetic
fields a particle sees will need to be interpolated from the grid
points. This is done by integrating the shape function in equation
\eqref{eq:single-particle}:
\begin{equation}
    \label{eq:field-interpolate}
    \mathbf{E}_p = \int d\mathbf{x}\: S(\mathbf{x} - \mathbf{x}_p)\mathbf{E}(\mathbf{x}) = \sum_c W(\mathbf{x}_c - \mathbf{x}_p)E_c
\end{equation}
where the subscript $c$ runs over all grid points. The function $W$ is
a weight function which has finite support centered around
$\mathbf{x}_p$ and sums to 1 on grid points where it does not
vanish. They are often chosen to be the so called \textit{B-spline}
functions, which are functions of minimal support with a given
polynomial degree. Higher order polynomial B-splines will in general
produce a smoother interpolation, but will be more computationally
expensive. In Aperture we support weight functions from 0-th order to
3-rd order polynomials.

Conversely, particle motion is interpolated onto the grid to produce a
current and charge density. To avoid spurious self-force, one needs to
use the same interpolation for both field to particle, and for
particle to current. We will discuss current deposition in greater
detail in section \ref{sec:charge-cons-curr}.

\subsection{Coordinate Systems}
\label{sec:coord-syst-sing}

Aperture supports the use of orthogonal curvilinear coordinate
systems. A coordinate system is given by three functions $h_1$, $h_2$,
and $h_3$ which are the scale functions in the three coordinate
directions.

The main coordinate systems we support are Cartesian, cylindrical,
spherical coordinates and some of their variants.

\subsection{Current Deposition}
\label{sec:charge-cons-curr}

We use the Esirkepov current deposition algorithm.

\subsection{Boundary Conditions}
\label{sec:boundary-conditions}

\subsection{Radiative Transfer}
\label{sec:radiative-transfer}


\section{Test Problems}
\label{sec:test-problems}
% This section we present some tests that show the validity of the
% code, in several coordinate systems, and in several physical scenarios

\section{Example Applications}
\label{sec:example-applications}
% This section we present some of toy physical models that we can
% simulate using this code, including axisymmetric pulsar (briefly),
% magnetar, twisted cylinder, Gruzinov problem, etc.

\section{Discussions and Remarks}
\label{sec:discussions}

% Local Variables:
% TeX-master: "../thesis"
% zotero-collection: #("16" 0 2 (name "Thesis"))
% End:
