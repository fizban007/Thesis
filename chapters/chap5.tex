\zotelo{../thesis.bib}

\chapter{Further Explorations of the Applications of PIC}
\label{chap:explorations}

\section{General Relativity}
\label{sec:general-relativity}

This is an exploratory effort of expanding a PIC code to support general
relativistic simulations. To my knowledge at the time of this writing, no such
code exists, and a successfully implemented code like this can open up the
possibility of simulating e.g.\ the pair discharge process around a black hole
% TODO: Observational implications?

\subsection{Metric and Observer}
\label{sec:metric}

This subsection is purely for reference. We record the results and move on. When
referring to a definition of a quantity one should check back to this subsection.
We start with the $3+1$ split following Macdonald \& Thorne (1982), writing the
metric as
\begin{equation}
  \label{eq:2}
  ds^2 = (\beta^2 - \alpha^2)dt^2 + 2\beta_idx^idt + \gamma_{ij}dx^idx^j
\end{equation}
where $\alpha$ is called the ``lapse function'' and $\beta$ is called the
``shift vector''. We consider only metric that are time independent meaning
$\partial_tg_{\alpha\beta} = 0$. We follow Komissarov (2004) in defining the
fiducial observer (FIDO) 4-velocity:
\begin{equation}
  \label{eq:4}
  n_{\mu} = (-\alpha, 0, 0, 0)
\end{equation}
The spatial metric tensor, which is used to raise and lower spatial indices, is
defined using this FIDO velocity
\begin{equation}
  \label{eq:1}
  \gamma_{\alpha\beta} = g_{\alpha\beta} + n_{\alpha}n_{\beta}
\end{equation}
Note since $n_{\alpha}$ only has the temporal component $\gamma_{ij}$ is the
same as $g_{ij}$. The other useful identities are:
\begin{align}
\label{eq:3}
n^{\mu} &= \frac{1}{\alpha}(1, -\beta^i) \\
g^{0\mu} &= -\frac{1}{\alpha}n^{\mu} \\
g^{ij} &= \gamma^{ij} - \frac{\beta^i\beta^j}{\alpha^2}
\end{align}
and
\begin{equation}
  \label{eq:5}
  g = -\alpha^{2}\gamma, \quad \beta^i = \gamma^{ij}\beta_{j}, \quad g = \mathrm{det} g_{\mu\nu}, \quad \gamma = \mathrm{det}\gamma_{ij}
\end{equation}

\subsection{Field Equations}
\label{sec:field}

We define as usual the Maxwell tensor from the vector potential
\begin{equation}
  \label{eq:6}
  F_{\mu\nu} = \partial_{\mu}A_{\nu} - \partial_{\nu}A_{\mu}
\end{equation}
where the physical meaning of $A_{\mu}$ will come clear later. In GR we are
supposed to replace $\partial_{\mu}$ with $\nabla_{\mu}$, but since
$\Gamma_{\alpha\beta}^{\mu}$ is symmetric in the lower indices, it is okay to
simply use partial derivative here. The upper $F^{\mu\nu}$ are defined simply by
raising the indices
\begin{equation}
  \label{eq:8}
  F^{\mu\nu} = g^{\mu\alpha}g^{\nu\beta}F_{\alpha\beta}
\end{equation}


The covariant Maxwell equations are the usual Maxwell equations with covariant
derivative in place of ordinary partial derivative:
\begin{equation}
  \label{eq:7}
  \nabla_{\nu}\tensor[^{*}]{F}{^{\mu\nu}} = 0, \quad \nabla_{\nu}F^{\mu\nu} = I^{\mu}
\end{equation}
where $I^{\mu}$ is the 4-current. Again, to associate physical meaning to this
current we need to relate it with particle motion, which we will do
subsequently.

Taking the first equation, the temporal and spatial parts read separately:
\begin{align}
  \frac{1}{\sqrt{-g}}\partial_i(\sqrt{-g}\tensor[^{*}]{F}{^{0i}}) &= 0 \\
  \frac{1}{\sqrt{-g}}\partial_t(\sqrt{-g}\tensor[^{*}]{F}{^{i0}}) + \frac{1}{\sqrt{-g}}\partial_j(\sqrt{-g}\tensor[^{*}]{F}{^{ij}}) &= 0
\end{align}
since $\sqrt{-g} = \alpha\sqrt{\gamma}$, we can write the spatial equation as:
\begin{equation}
  \label{eq:10}
  \frac{1}{\sqrt{\gamma}}\partial_t(\sqrt{\gamma}\alpha\tensor[^{*}]{F}{^{i0}}) + \frac{1}{\sqrt{\gamma}}\partial_j(\alpha\sqrt{\gamma}\tensor[^{*}]{F}{^{ij}}) = 0
\end{equation}
This can be simplified to something we can identify with, by defining new fields
$B^i = \alpha\tensor[^{*}]{F}{^{i0}}$ and $E_i = \frac{1}{2}\alpha
e_{ijk}\tensor[^{*}]{F}{^{jk}}$ (we also used the fact that $\sqrt{\gamma}$ is time-independent):
\begin{equation}
  \label{eq:11}
  \partial_tB^i + e^{ijk}\partial_jE_k = 0
\end{equation}
where $e_{ijk} = \sqrt{\gamma}\epsilon_{ijk}$ is the Levi-Civita pseudo-tensor
and $\epsilon_{ijk}$ is the totally antisymmetric symbol which is either 1 or
$-1$. Similarly $e^{ijk} = \epsilon^{ijk}/\sqrt{\gamma}$ where $\epsilon^{ijk}$
is numerically the same as $\epsilon_{ijk}$. This equation looks like the usual
Maxwell equation where $\partial_t\mathbf{B} = -\nabla\times \mathbf{E}$,
however we try not to rely on the usual curl and bold-face vector notation, and
make clear whenever we talk about a field component whether it is the upper or
lower index version.

The spatial part of the second Maxwell equation reads:
\begin{equation}
  \label{eq:14}
  \frac{1}{\sqrt{\gamma}}\partial_t(\sqrt{\gamma}\alpha\tensor{F}{^{i0}}) + \frac{1}{\sqrt{\gamma}}\partial_j(\alpha\sqrt{\gamma}\tensor{F}{^{ij}}) = \alpha I^{i}
\end{equation}
Similarly we can identify new fields $D^i = \alpha F^{0i}$ and $H_i =
\frac{1}{2}\alpha e_{ijk}F^{jk}$, such that this equation reads:
\begin{equation}
  \label{eq:15}
  -\partial_tD^i + e^{ijk}\partial_jH_k = \alpha I^i
\end{equation}

There are in total of 4 dynamic fields: $B^i$, $E_i$, $D^i$, and $H_i$. They are
not independent so we want to find their relations to cut down the number of
fields. From the definition of the Maxwell field tensor we have in vacuum
\begin{equation}
  \label{eq:17}
  \tensor[^{*}]{F}{^{\alpha\beta}} = \frac{1}{2}e^{\alpha\beta\mu\nu}F_{\mu\nu}, \quad F^{\mu\nu} = \frac{1}{2}e^{\mu\nu\alpha\beta}\tensor[^{*}]{F}{_{\alpha\beta}}
\end{equation}
so we can find that
\begin{equation}
  \label{eq:18}
  H_i = \tensor[^{*}]{F}{_{0i}}, \quad E_i = F_{i0}, \quad B^i = \frac{1}{2}e^{ijk}F_{jk}, \quad D^i = \frac{1}{2}e^{ijk}\tensor[^{*}]{F}{_{jk}}
\end{equation}
To figure out the relations between them we can take:
\begin{equation}
  \label{eq:19}
  \begin{split}
  \alpha D^i &= \alpha^2F^{0i} = \alpha^2g^{0\alpha}g^{i\beta}F_{\alpha\beta} \\
             &= \alpha^2g^{00}g^{ij}F_{0j} + \alpha^2g^{0j}g^{i0}F_{j0} + \alpha^2g^{0j}g^{ik}F_{jk} \\
             &= \gamma^{ij}E_j - \frac{\beta^i\beta^j}{\alpha^2}E_j + \frac{\beta^i\beta^j}{\alpha^2}E_j + \beta^jg^{ik}F_{jk} \\
             &= \gamma^{ij}E_j + \beta^j\gamma^{ik}e_{jkl}B^l - \frac{1}{\alpha}\beta^j\beta^i\beta^ke_{jkl}B^l
  \end{split}
\end{equation}
Since the last term on the left hand side is zero due to antisymmetry we have:
\begin{equation}
  \label{eq:20}
  E_i = \alpha \gamma_{ij}D^j + e_{ijk}\beta^jB^k
\end{equation}
Similarly we can work from $\alpha B^i$ to get
\begin{equation}
  \label{eq:21}
  H_i = \alpha \gamma_{ij}B^i - e_{ijk}\beta^jD^k
\end{equation}

Since we can express $E_i$ and $H_i$ in terms of $B^i$ and $D^i$ and some metric
coefficients, we will call $B^i$ and $D^i$ our dynamic fields. It is also
convenient that equations \eqref{eq:11} and \eqref{eq:15} are already time
evolution equations of these two dynamic fields. Therefore our general strategy
of solving them will be to figure out the auxiliary fields $E_i$ and $H_i$ from
equations \eqref{eq:20} and \eqref{eq:21}, then use them to update the fields
$B$ and $D$.

It is also worth writing down the integral version of the equations
\eqref{eq:11} and \eqref{eq:15}. If $\Sigma$ is the cell face in direction $i$,
then we have
\begin{align}
  \label{eq:22}
  \partial_t\int_{\Sigma}\frac{1}{2}e_{ijk}D^i\,dx^j\wedge dx^k &= \oint_{\partial \Sigma}H_i\,dx^i - \int_{\Sigma}\frac{1}{2}\alpha e_{ijk}I^i\,dx^j\wedge dx^k \\
  \partial_t\int_{\Sigma}\frac{1}{2}e_{ijk}B^i\,dx^j\wedge dx^k &= -\oint_{\partial \Sigma}E_i\,dx^i
\end{align}

\subsection{Particle Equations}
\label{sec:particles}

Now that we decided that $B^i$ and $D^i$ should be our dynamic fields, we need
to find how they affect particles. To do that we use the Lagrangian formalism
and write down the action as follows (Landau \& Lifshitz):
\begin{equation}
  \label{eq:23}
  S = \sum\int mc d\tau - \sum \int \frac{e}{c}A_{\mu}\,dx^{\mu} - \frac{1}{16\pi c}\int F_{\mu\nu}F^{\mu\nu}\,\sqrt{-g}d\Omega
\end{equation}
where the sum is over all particles, and $d\Omega$ is the spacetime volume element.

We try to follow Landau \& Lifshitz to find the equation of motion for a single
particle. The principle of least action states that (for the two terms of the
action):
\begin{equation}
  \label{eq:9}
  \delta S = \delta \int \left(mc\,d\tau - \frac{e}{c} A_{\mu}\,dx^{\mu}\right) = 0
\end{equation}
We do the variation of the first part in the following way. Starting with
$d\tau^2 = -ds^2 = -g_{\mu\nu}dx^{\mu}dx^{\nu}$, we have
\begin{equation}
  \label{eq:12}
  \delta d\tau^2 = 2d\tau \delta d\tau = -\delta(g_{\mu\nu}dx^{\mu}dx^{\nu}) = -dx^{\mu}dx^{\nu}\frac{\partial g_{\mu\nu}}{\partial x^{\sigma}}\delta x^{\sigma} - 2g_{\mu\nu}dx^{\mu}d \delta x^{\nu}
\end{equation}
we can then write the variation of the whole action
\begin{equation}
  \label{eq:13}
  \begin{split}
    \delta S &= -mc\int \left( \frac{1}{2}\frac{dx^{\mu}}{d\tau}\frac{dx^{\nu}}{d\tau}\partial_{\sigma}g_{\mu\nu}\delta x^{\sigma} + g_{\mu\nu}\frac{dx^{\mu}}{d\tau}\frac{d\delta x^{\nu}}{d\tau} \right) d\tau - \frac{e}{c}\int \left( A_{\mu}d\delta x^{\mu} + \delta A_{\mu} dx^{\mu} \right) \\
    &= -mc\int \left[ \frac{1}{2}\frac{dx^{\mu}}{d\tau}\frac{dx^{\nu}}{d\tau}\partial_{\sigma}g_{\mu\nu}\delta x^{\sigma} - \frac{d}{d\tau}\left( g_{\mu\nu}\frac{dx^{\mu}}{d\tau} \right) \delta x^{\nu} \right]d\tau - \frac{e}{c}\int \left( \partial_{\sigma}A_{\mu}dx^{\mu}\delta x^{\sigma} - \partial_{\sigma}A_{\mu}dx^{\sigma}\delta x^{\mu} \right) \\
    &= -mc\int \left[ \frac{1}{2}\frac{dx^{\mu}}{d\tau}\frac{dx^{\nu}}{d\tau}\partial_{\sigma}g_{\mu\nu} - \frac{d}{d\tau}\left( g_{\mu\sigma}\frac{dx^{\mu}}{d\tau} \right) \right]\delta x^{\sigma} d\tau - \frac{e}{c}\int (\partial_{\sigma}A_{\mu} - \partial_{\mu}A_{\sigma})\frac{dx^{\mu}}{d\tau}\delta x^{\sigma}d\tau
  \end{split}
\end{equation}
Since the variation should vanish for any $\delta x^{\sigma}$, the integrand
should vanish therefore we get our dynamic equation (as expected):
\begin{equation}
  \label{eq:16}
  g_{\mu\sigma}\frac{du^{\mu}}{d\tau} + \frac{1}{2}\left( \partial_{\nu}g_{\mu\sigma} + \partial_{\mu}g_{\nu\sigma} - \partial_{\sigma}g_{\mu\nu} \right)u^{\mu}u^{\nu} = \frac{e}{c}(\partial_{\sigma}A_{\mu} - \partial_{\mu}A_{\sigma})u^{\mu}
\end{equation}
where $u^{\mu} = dx^{\mu}/d\tau$ is the 4-velocity of the particle. We shall
always use $u^{\mu}$ to denote the standard 4-velocity from now on. In other
words we have
\begin{equation}
  \label{eq:24}
  mc\frac{Du^{\mu}}{D\tau} = \frac{e}{c}F^{\mu\nu}u_{\nu}, \quad \text{or} \quad mc\frac{D u_{\mu}}{D\tau} = \frac{e}{c}F_{\mu\nu}u^{\nu}
\end{equation}
These two are equivalent because the covariant derivative defined by Christoffel
symbols are automatically metric-compatible, so $\nabla_{\sigma}g_{\mu\nu} = 0$.
We can therefore freely raise and lower indices inside covariant derivatives.

If we take the spatial part of the second equation, and multiply it by
$d\tau/dt$, we get (from now on we ignore factors of $c$)
\begin{equation}
  \label{eq:25}
  m\frac{Du_i}{D\tau}\frac{d\tau}{dt} = eF_{i\nu}u^{\nu}\frac{d\tau}{dt},\quad \Longrightarrow m\frac{Du_i}{Dt} = eF_{i\nu}\frac{dx^{\nu}}{dt}
\end{equation}
For simplicity of notation lets use $Du_i/Dt$ to refer to the product
$(Du_i/D\tau)(d\tau/dt)$ although it might not be a proper covariant derivative.
We also define $v^{\mu} = dx^{\mu}/dt$. From here on $u^{\mu}$ will mean the
ordinary 4-velocity and $v^{\mu}$ means the derivative with respect to
coordinate time. Note $v^0 = 1$ by definition. We can now write
\begin{equation}
  \label{eq:26}
  \begin{split}
  m\frac{Du_i}{Dt} &= e(F_{i0}v^0 + F_{ij}v^j) = e(E_i + e_{ijk}v^jB^k) \\
                   &= e \left( \alpha \gamma_{ij}D^j + e_{ijk}\beta^jB^k + e_{ijk}v^jB^k \right)
  \end{split}
\end{equation}
The last line is because we want to express particle acceleration in terms of
our dynamic fields $B^i$ and $D^{i}$. Due to the form of the equation, it is
convenient to define a new velocity $\bar{v}$ such that
\begin{equation}
  \label{eq:27}
  \alpha \bar{v}^{i} - \beta^i = v^i
\end{equation}
Thus we will have
\begin{equation}
  \label{eq:28}
  m\frac{Du_i}{Dt} = e\alpha(\gamma_{ij}D^j + e_{ijk}\bar{v}^jB^k)
\end{equation}
which looks like the ordinary Lorentz force.

Now the remaining problem is that we only have an update equation for $u_i$, so
we need a relation between $u_i$ and $\bar{v}^{i}$. This is relatively simple
since we know $u^{\mu}u_{\mu} = -1$ for a time-like worldline. Let's call $u^0 =
dt/d\tau = \gamma_p$. Then
\begin{equation}
  \label{eq:29}
  u_i = g_{i\nu}u^{\nu} = \beta_i\gamma_p + \gamma_{ij}\gamma_pv^j = \alpha \gamma_p \gamma_{ij}\bar{v}^j
\end{equation}
And the particle Lorentz factor $\gamma_p$ is related to $\bar{v}^i$ by (from
$u^{\mu}u_{\mu} = -1$)
\begin{equation}
  \label{eq:30}
  \gamma_{p}^2 = \frac{1}{\alpha^2(1 - \gamma_{ij}\bar{v}^i\bar{v}^j)}
\end{equation}
So the Lorentz factor is similar to conventional one with $\bar{v}$, except for
the extra factor of $\alpha$.

Because by definition we have $v^i = dx^i/dt$, the particle positions can be
updated by
\begin{equation}
  \label{eq:31}
  \frac{dx^i}{dt} = v^i = \alpha \bar{v}^i - \beta^{i}
\end{equation}

The only remaining problem with particle dynamics is that we have a hard time
evaluating the covariant derivative $Du_i/Dt$. An approximation might be simply
replace it with $du_i/dt$. We will not go into possible different schemes here.

Additional note on implementation of the particle pusher. We need to find an
efficient way to store the quantities that we need, which in this case is
$\bar{v}^i$. However when the particles become very relativistic, $\bar{v}^i$ is
close to 1 and therefore prone to numerical truncation error. It's much better
to store some form of momentum. We could store directly $u_i$ which we update,
but then it is slightly difficult to find $\gamma_p$ from it, where a matrix
inversion of $\gamma_{ij}$ is involved. As a compromise, we can define
\begin{equation}
  \label{eq:38}
  \bar{p}^i = \alpha\gamma_p\bar{v}^i
\end{equation}
so that we have the following relations:
\begin{equation}
  \label{eq:39}
  u_i = \gamma_{ij}\bar{p}^j,\quad \gamma_p^2 = \frac{1}{\alpha^2}(1 + \gamma_{ij}\bar{p}^i\bar{p}^j)
\end{equation}

The particle momentum update step now should look like the following: First we
find $u_i$ using equation \eqref{eq:39}, and $\bar{v}^i$ using equation
\eqref{eq:38}, then we use equation \eqref{eq:28} and the modified Vay algorithm
\ref{sec:Vay-mod} to find $\Delta u_i$ over the time step. When we have the new
$u_i$ we can invert back to find the $\bar{p}^i$ again by inverting the matrix
$\gamma_{ij}$. This step will be simple in metrics where $\gamma_{ij}$ is
diagonal. If we store $\bar{p}^i$ as the momentum of the particles, then
$\bar{v}^i$ can simply be calculated by dividing it by $\alpha\gamma_p$. It
might be useful to store $\alpha\gamma_p$ of the particle (where $\alpha$ is
evaluated at the particle position) in the particle storage as well.

\subsection{Modified Vay Algorithm}
\label{sec:Vay-mod}

% Vay (2008)
\citet{vay_simulation_2008}
showed a stable algorithm for pushing particles in electromagnetic
field. It was designed to solve discretized equation
\begin{equation}
  \label{eq:40}
  \frac{\gamma^{n+1}\mathbf{v}^{n+1} - \gamma^n\mathbf{v}^n}{\Delta t} = \frac{q}{m}\left( \mathbf{E}^{n+1/2} + \frac{\mathbf{v}^n + \mathbf{v}^{n+1}}{2}\times \mathbf{B}^{n+1/2} \right)
\end{equation}
where $\gamma^n = (1 - \mathbf{v}^n\cdot \mathbf{v}^n)^{-1/2}$. Notice that we
only have one kind of vector in this equation, namely $\mathbf{v}$, or in our
notation, the contravariant vector $v^i$. If we discretize our equations in the
same way, then we will find the following equation:
\begin{equation}
  \label{eq:41}
  \frac{\gamma_{ij}(\bar{p}^j)^{n+1} - \gamma_{ij}(\bar{p}^j)^n}{\Delta t} = \frac{\alpha q}{m}\left( \gamma_{ij}D^j + \sqrt{\gamma}\epsilon_{ijk}\frac{(\bar{v}^j)^n + (\bar{v}^j)^{n+1}}{2}B^k \right)
\end{equation}
where $\bar{p}^i = \alpha\gamma_p\bar{v}^i$ is similar to how usual relativistic
momentum is related to velocity. The crucial difference from \eqref{eq:40} is
the appearance of the metric matrix $\gamma_{ij}$ on the left hand side.

The usual Vay algorithm solves equation \eqref{eq:40} using two steps. First
step is to split the equation into two parts and define (where $\mathbf{u} =
\gamma \mathbf{v}$)
\begin{equation}
  \label{eq:42}
  \mathbf{u}' = \mathbf{u}^n + \frac{q\Delta t}{m}\left( \mathbf{E}^{n+1/2} + \frac{\mathbf{v}^n}{2}\times \mathbf{B}^{n+1/2} \right)
\end{equation}
then one solve the resulting implicit equation
\begin{equation}
  \label{eq:43}
  \mathbf{u}^{n+1} = \mathbf{u}' + \frac{q\Delta t}{m}\left( \frac{\mathbf{u}^{n+1}}{2\gamma^{n+1}}\times \mathbf{B}^{n+1/2} \right)
\end{equation}
If one writes $\mathbf{b} = (q\Delta t/2m)\mathbf{B}^{n+1/2}$, then
the solution to the above equation is simply (writing $\gamma$ instead of
$\gamma^{n+1}$ for simplicity)
\begin{equation}
  \label{eq:44}
  \mathbf{u}^{n+1} = \left[ \mathbf{u}' + (\mathbf{u}'\cdot \mathbf{b})\mathbf{b}/\gamma^2 + \mathbf{u}'\times \mathbf{b}/\gamma \right] / (1 + b^2/\gamma^2)
\end{equation}
This equation is still incomplete because $\gamma^{n+1}$ appears on the right
hand side, so we need to carry out the second step, which is to solve for
$\gamma$. We can do that by dotting $\mathbf{u}^{n+1}$ to equation \eqref{eq:43}
and make use of the definition of $\gamma$, $\gamma = \sqrt{1 + u^2}$:
\begin{equation}
  \label{eq:45}
  \gamma^2 = 1 + u^2 = 1 + \mathbf{u}'\cdot \mathbf{u} = 1 + \frac{u'^2 + (\mathbf{u}'\cdot \mathbf{b})^2/\gamma^2}{1 + b^2/\gamma^2}
\end{equation}
This is a quadratic equation for $\gamma^2$ and we can solve it to find
$\gamma$. Finally one can plug this back into equation \eqref{eq:44} to find the
updated momentum $\mathbf{u}^{n+1}$.

In our GR case things becomes more complicated. Due to the appearance of
$\gamma_{ij}$ matrix in equation \eqref{eq:41}, the direction solution to the
equation \eqref{eq:44} doesn't work anymore. We still define the intermediate
$u'$ as
\begin{equation}
  \label{eq:47}
  u'_i = u_i^n + \frac{\alpha q\Delta t}{m}\left(\gamma_{ij}D^j + \sqrt{\gamma}\epsilon_{ijk}\frac{\bar{v}^j}{2}B^k\right)
\end{equation}
Now one needs to solve the following equation:
\begin{equation}
  \label{eq:46}
  \left( \gamma_{ij} - \frac{\epsilon_{ijk}b^k}{\alpha\gamma_p} \right)\bar{p}^j = u'_i
\end{equation}
where $b^k = (\alpha\sqrt{\gamma} q \Delta t/2m)B^{k}$. The difference is that
$\gamma_{ij}$ would have been $\delta_{ij}$ in the case of flat space. However,
this is still a $3\times 3$ matrix equation and one can invert it by brute
force.

Assuming it is done, we can plug the equation into the relation of $\gamma_p$
with $\bar{p}^i$ which is
\begin{equation}
  \label{eq:48}
  \alpha^2\gamma_p^2 = 1 + \gamma_{ij}\bar{p}^i\bar{p}^j = 1 + \bar{p}^iu'_i = 1 + \left( \gamma_{ij} - \frac{\epsilon_{ijk}b^k}{\alpha\gamma_p} \right)^{-1}u'_iu'_j
\end{equation}
At this point it's obvious we should call the combination $\alpha\gamma_p$ as
the convenient Lorentz factor $\bar{\gamma}_p$. Again this turns out to be a
quadratic equation for $\bar{\gamma}_p$, albeit more complicated. I plugged the
matrix into Mathematica to find the full inverted result (which is surprisingly
not that complicated):
\begin{equation}
  \label{eq:49}
  \bar{\gamma}_p^2 = 1 + \frac{(u'_ib^i)^2 + \bar{\gamma}_p^2\sigma}{\bar{\gamma}_p^2\det \gamma + \gamma_{ij}b^ib^j}
\end{equation}
where the symbol $\sigma$ is (assuming $\gamma_{ij}$ is symmetric, which is
always true for a metric):
\begin{equation}
  \label{eq:50}
\begin{split}
  \sigma &= u'^2_{1}(\gamma_{22}\gamma_{33} - \gamma_{23}^2) + u'^2_2(\gamma_{11}\gamma_{33}-\gamma_{13}^2) + u'^2_3(\gamma_{11}\gamma_{22} - \gamma_{12}^2) \\
& + 2u'_1u'_2(\gamma_{13}\gamma_{23} - \gamma_{12}\gamma_{33}) + 2u'_1u'_3(\gamma_{12}\gamma_{32} - \gamma_{13}\gamma_{22}) + 2u'_2u'_3(\gamma_{21}\gamma_{31} - \gamma_{23}\gamma_{11})
\end{split}
\end{equation}
It would be nice to simplify this $\sigma$ term further! However even at this
form one should be able to code it into an algorithm. After one solves equation
\eqref{eq:49} for $\bar{\gamma}_p$ one can plug it into equation \eqref{eq:46}
to solve for $\bar{p}^i$. This is the modified Vay algorithm for curved space.

\subsection{Effect of Gravity}
\label{sec:gravity}

Since we have written our particle push equation in complete form:
\begin{equation}
  \label{eq:51}
  m\frac{Du_i}{Dt} = e\alpha (\gamma_{ij}D^j + e_{ijk}\bar{v}^jB^k)
\end{equation}
it would be nice to solve it completely accounting for GR effect as well! I
think it can be done. The covariant derivative on the left hand side can be
expanded as
\begin{equation}
  \label{eq:52}
  \frac{Du_i}{D\tau} = \frac{du_i}{d\tau} - \Gamma^{\alpha}_{i\nu}u_{\alpha}u^{\nu} = \frac{du_i}{dt} - \frac{1}{2}\left( g_{i\alpha,\mu} + g_{\mu\alpha,i} - g_{i\mu,\alpha} \right)u^{\alpha}u^{\mu}
\end{equation}
Two terms in the bracket are antisymmetric with respect to $\alpha$ and $\mu$
whereas $u^{\alpha}u^{\mu}$ is symmetric, so we only have
\begin{equation}
  \label{eq:53}
  \frac{Du_i}{D\tau} = \frac{du_i}{d\tau} - \frac{1}{2}g_{\mu\alpha,i}u^{\mu}u^{\alpha}
\end{equation}
and if we convert the time to coordinate time we have
\begin{equation}
  \label{eq:54}
  \frac{Du_i}{Dt} = \frac{du_i}{dt} - \frac{1}{2}g_{\mu\alpha,i}v^{\alpha}g^{\mu\nu}u_{\nu}
\end{equation}
The inverse metric $g^{\mu\nu}$ was inserted so that the right hand side also
has lower indices $u_{\nu}$. One can totally pre-store the coefficients
$g_{\mu\alpha,i}g^{\mu\nu}/2$ on the grid, then convolve with $v^{\alpha}$ to
get the matrix in front of $u_{\nu}$.

Now to combine the Lorentz force with the gravitational correction, one can do
the so-called ``drift-kick'' trick. The idea is that, we want to separate a
single timestep into two half-steps. In the first half-step, the particle simply
drifts along the geodesic, and we have only
\begin{equation}
  \label{eq:55}
  \frac{u_i^{n+1/2} - u_i^n}{\Delta t/2} = \frac{1}{2}g_{\mu\alpha,i}v^{\alpha}g^{\mu\nu}u_{\nu}
\end{equation}
Since the right hand side is quadratic in $u$, we either use fully explicit
scheme, evaluating the right hand side at time step $n$, or we use a
``semi-implicit'' scheme and solve the following matrix equation
\begin{equation}
  \label{eq:56}
  \frac{u_{\mu}^{n+1/2} - u_{\mu}^n}{\Delta t/2} = \frac{1}{2}g_{\alpha\beta,\mu}(v^{\alpha})^ng^{\beta\nu}(u_{\nu})^{n+1/2}
\end{equation}
Note that to do this we should save $u_0$ as part of the dynamic variables now.

After we ``drifted'' the particles for a half-step, we evaluate the ``kick'' on
the particle due to electromagnetic force. This is computed using the modified
Vay algorithm detailed in the above subsection. We solve the following equation
\begin{equation}
  \label{eq:57}
  \frac{u_i' - u_i}{\Delta t} = \frac{\alpha q}{m}\left( \gamma_{ij}D^j + e_{ijk}\frac{\bar{v}'^j + \bar{v}^j}{2}B^k \right)
\end{equation}
where the primed quantities are the momentum and velocity after the kick. One
can then compute $u_0'$ from $u_i'$ using the constraint $u_{\mu}u^{\mu} = -1$.
We have:
\begin{equation}
  \label{eq:59}
  u_0 = \alpha\gamma_p \left[ \gamma_{ij}\beta^i\bar{v}^j - \alpha \right]
\end{equation}

Then we use these quantities to drift the particles for another half-step
\begin{equation}
  \label{eq:58}
  \frac{u_{\mu}^{n+1} - u'^{n+1/2}_{\mu}}{\Delta t/2} = \frac{1}{2}g_{\alpha\beta,\mu}(v'^{\alpha})^{n+1/2}g^{\beta\nu}(u_{\nu})^{n+1}
\end{equation}

One question that may be asked is that is the energy of the particle conserved
during the update. This is on the TODO list and we might go back to this problem
a bit later.

\subsection{Current Deposition}
\label{sec:current}

One needs to define the current $I^{\mu}$ that occurs in the Maxwell equations
\eqref{eq:7} in terms of particle motion. Again following Landau \& Lifshitz we
do that by referring to the action \eqref{eq:23}. We introduce the charge
density $\rho$ by summing the charge contained in a certain spatial volume element
\begin{equation}
  \label{eq:32}
  de = \rho\sqrt{\gamma}dx^1dx^2dx^3
\end{equation}
Multiplying this by $dx^i$ we have
\begin{equation}
  \label{eq:32}
  de dx^i = \rho dx^\mu\sqrt{\gamma}dx^1dx^2dx^3 = \frac{\rho}{\alpha}\sqrt{-g}d\Omega\,\frac{dx^\mu}{dt}
\end{equation}
where $d\Omega = dtdx^1dx^2dx^3$. We therefore define the current 4-vector as
\begin{equation}
  \label{eq:33}
  I^{\mu} = \frac{\rho c}{\alpha}\frac{dx^{\mu}}{dt}
\end{equation}

Charged particles are represented by delta function in the current and charge
distributions. Let's define $\int \delta(\mathbf{r})d^3x = 1$, then the charge
and current densities (Landau \& Lifshitz) are
\begin{equation}
  \label{eq:34}
  \rho = \sum_p\frac{q_p}{\sqrt{\gamma}}\delta(\mathbf{r}-\mathbf{r}_p),\quad I^{\mu} = \sum_p\frac{q_pc}{\alpha\sqrt{\gamma}}\delta(\mathbf{r} - \mathbf{r}_p)\frac{dx^{\mu}}{dt}
\end{equation}
The continuity equation is $\nabla_{\mu}I^{\mu} = 0$ and using the divergence
form \eqref{eq:35} and the above definition we can see it is indeed satisfied.
The form therefore suggests that in depositing the current the correct
velocities we should use are $v^i$, definitely not $\bar{v}^i$.

We can also find the integral form of the continuity equation. Since we have
\begin{equation}
  \label{eq:35}
  \frac{1}{\sqrt{-g}}\partial_{\mu}\left( \sqrt{-g} I^{\mu} \right) = 0
\end{equation}
we can separate into spatial and temporal part, and integrate both over a given
cell:
\begin{equation}
  \label{eq:36}
  \partial_t\int_C \alpha\sqrt{\gamma}I^0\,d^3x + \int_C \partial_i \left( \alpha\sqrt{\gamma}I^i \right)\,d^3x = 0
\end{equation}
We can use Gauss's theorem to replace the second integral. Because $\int_{\Sigma} d\omega
= \oint_{\partial\Sigma}\omega$ we have
\begin{equation}
  \label{eq:37}
  \partial_t\int_Cc\rho\sqrt{\gamma}\,d^3x + \oint_{\partial C}\frac{1}{2}e_{ijk}\alpha I^i\,dx^j\wedge dx^k = 0
\end{equation}
Notice that the second integral is exactly the integral that occurs in the
integral form of the Maxwell equations \eqref{eq:22}. Therefore if using
integral formalism, the we only need to separate the change of charge in a given
cell into different directions, find the current flux easily and then directly
use them to update the dynamic fields $B^i$ and $D^i$.

\section{Unstructured Grid}
\label{sec:unstructured-pic}



% Local Variables:
% TeX-master: "../thesis"
% zotero-collection: #("16" 0 2 (name "Thesis"))
% End: