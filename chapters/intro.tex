\zotelo{../thesis.bib}

\chapter{Introduction}
\label{chap:intro}

A lot of astrophysics has to do with observing objects at various radiation
frequencies. Many objects are sources of high energy radiation, including but
not limited to pulsars, gamma-ray bursts, magnetars, super-massive black holes,
etc. In many of the objects, the spectrum is highly non-thermal, and the
radiation is produced by accelerated particles. The study of particle
acceleration and dissipation of other forms of energy into particle kinetic
energy is therefore vital in the study of high-energy astrophysics.

In many of the sources, the most notable source of energy for nonthermal
particles is the magnetic energy. In some sources such as pulsars, magnetic
field plays an intermediary role, acting as a channel that converts the
rotational energy of the pulsar ultimately into the particle energy, which then
is radiated away and produce the observed radio/X-ray/gamma ray emission. In
other sources such as magnetars it is the magnetic energy itself that is
directly converted to particle energy that is radiated.

% TODO: Go into more details, and mention past work
The dissipation of magnetic energy and acceleration of particles is a highly
nonlinear process, and very difficult to model directly. In the past people have
tried different methods. Some attempt to model the whole system by treating the
plasma as fluid, and use magnetohydrodynamics or force-free approximation. Some
isolate part of the system and study the detailed local evolution of
eletromagnetic fields. It is only recently that computational power has
increased to the level that we could attempt a first-principle direct simulation
of the system of interest.

This dissertation will be focusing on the physics of isolated neutron stars. In
the following sections we will examine the history and basic physics of
rotation-powered pulsars, and the exotic magnetars which have extremely high
magnetic field. Finally we will outline the chapters of the thesis.

\section{Pulsars}
\label{sec:intro-pulsars}

\subsection{Observations}

\subsection{Theoretical models and problems}


\section{Magnetars}
\label{sec:intro-magnetars}

\subsection{Observations}

\subsection{Theoretical Studies}


\section{This Dissertation}
\label{sec:intro-outline}

In this dissertation we will attempt to address the problem of particle
acceleration and global structure of the magnetosphere of pulsars and
magnetars.

Chapter \ref{chap:pic} will be devoted to a detailed introduction of the
particle-in-cell technique, which will be the basic numerical tool for our
study.

Chapter \ref{chap:polar-cap} will focus on a local study of the pulsar
polar cap. We will look at the region well within the pulsar magnetosphere, and
approximate the geometry as 1D. We will discuss implications of this
approximation, and what we can and can not learn from this local study.

Chapter \ref{chap:pulsar} will study the global pulsar magnetosphere, motivated
by the study of the polar cap particle acceleration.

Chapter \ref{chap:magnetar} will study the twisted magnetosphere of magnetars.
